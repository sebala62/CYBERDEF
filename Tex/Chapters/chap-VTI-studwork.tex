


\section{Travaux personnels}

\subsection{généralités}

% Begin PRZ ===========================
\begin{frame}
\frametitle<presentation>{Votre travail}
Dans le cadre de ce cours, un seul travail  est demandé. C'est un travail personnel, dont l'objectif est de vous faire travailler sur un sujet que vous souhaiter étudier dans le but de le présenter aux autres. Vous pouvez donc choisir un sujet que vous maitrisez ou un sujet que vous ferez découvrir avec un regard de béotien.
\end{frame}
% end PRZ ========================

En résumé votre travail devra être : 

% Begin PRZ ===========================
\begin{frame}
\frametitle<presentation>{Votre travail}
% end header PRZ =======================
\begin{itemize}
  \item  1 document de moins de 20 pages
  \item  Sur un produit du monde de la Sécurité Opérationnelle
  \item  Un travail de votre expérience, ou simplement sur une recherche sur internet pour un produit à choisir …
\end{itemize}


\end{frame}
% end PRZ ========================

Votre  analyse sera étayée et critique sur un élément de la sécurité opérationnelle. La notion d'élément SECOPS regroupe de nombreuses thématiques :
% Begin PRZ ===========================
\begin{frame}
\frametitle<presentation>{Thèmes}
% end header PRZ =======================
\begin{itemize}
  \item Méthodologique;
  \item Technologique ou technique;
  \item Conceptuel;
  \item Juridique...
\end{itemize}

\end{frame}
% end PRZ ========================



Sur ces thématiques, il est important que votre sujet de FICHE TECHNO reste dans le domaine de la sécurité opérationnelle :

% Begin PRZ ===========================
\begin{frame}
\frametitle<presentation>{Domaines}
% end header PRZ =======================
\begin{itemize}
  \item \textbf{VEILLE/AUDIT} : Des produits/services de veille et de scan de vulnérabilités informatiques (Qualys, nessus, nmap, checkmarx, appscan … et bien d’autres …) 
  \item \textbf{SURVEILLE/ALERTE} Des produits/services de gestion d’événement, de supervision et d’alerte (Log, SIEM : Qradar, ArcSight, LogPoint, splunk … et bien d’autres …)
  \item \textbf{ANALYSE/REPONSE} : Des produits/services d’analyse post-mortem, et de forensique (Forensic Toolkit, encase … et bien d’autres …)
\end{itemize}

\end{frame}
% end PRZ ========================

Votre travail est à rendre en fin de session, sous forme informatique (OpenDoc, Formats Microsoft, PDF, Latex...)

\subsection{méthode de notation}

Votre travail est noter sur différents critères ci dessous.



Chaque critère est évalué suivant les valeurs suivantes

% Begin PRZ ===========================
\begin{frame}
\frametitle<presentation>{Critères}
% end header PRZ =======================

\begin{itemize}
  \item Qualité du positionnement du problème ou du sujet
  \item Qualité de la conclusion, dont l'ouverture vers d'autres points
  \item Présence et affichage de votre point de vue : Apports personnels : apports liés à sa propre expérience
\end{itemize}

\end{frame}
% end PRZ ========================
Les valeurs d'évaluation de ces critères sont :

% Begin PRZ ===========================
\begin{frame}
\frametitle<presentation>{Evaluation}
% end header PRZ =======================

\begin{itemize}
  \item 0 - Travaux trop simpliste et sans valeur d'apport personnel;
  \item 1 - Travaux simple ou sans apport personnel;
  \item 2 - Apport étayé et présentation claire;
  \item 4 - Apport didactique;
  \item 5 - Apport personnel étayé.
\end{itemize}
\end{frame}
% end PRZ ========================

\subsection{Format}

Si le format n'est pas imposé, il est demandé toutefois de suivre un plan permettant de suivre votre démarche et permettant d'être le plus pédagogique possible.
Vous pouvez utiliser le modèle de document mis à votre disposition.

\begin{itemize}
  \item WORD : SEC101-Part3-Modele-Fiche-Techno-VxRy
  \item Latex :  MemModel sur GITHUB (Cyberdef101)
\end{itemize}

\section{Sujets}

\subsection{Sélection des sujets}

% Begin PRZ ===========================
\begin{frame}
\frametitle<presentation>{Travaux à valider}
Avant de vous lancer dans vos travaux, il est demander de faire valider votre sujet par l'enseignant. Pour cela simplement envoyer un mail avec votre sujet et vos justificatif de choix.

Vous trouverez ci après quelques différentes thématiques avec des idées de sujet. Chaque sujet est constitué d'un thème, et d'un descriptif optionnel.
Ces sujets sont donnés à titre indicatif. Il vous revient d'en proposer un si aucun de ceux présentés vous intéressent.

Votre travail est \textbf{à rendre} en fin de session

\end{frame}
% end PRZ ========================

%*********************************************
\subsection{Sujets : Vulnerability Management}
%----------------------------------------------------
\subsubsection{Exemples étayés de vulnérabilités}
\subsubsection{BugBounty}
\subsubsection{Outils au service des tests d'intrusion}



%*********************************************
\subsection{Sujets : Threat Management}
%----------------------------------------------------
\subsubsection{Description d'une attaque virale}
\subsubsection{Architecture d'un BOTNET}
\subsubsection{Description attaque DDOS}


%*********************************************
\subsection{Sujets : Incident Management}
%----------------------------------------------------
\subsubsection{Description attaque DDOS}


%*********************************************
\subsection{Sujets : Crisis Management}
%----------------------------------------------------
\subsubsection{ISO22301}
\subsubsection{Annuaire de crise}





