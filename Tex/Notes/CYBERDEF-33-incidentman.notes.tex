%========================================
%. CONFIGURATION ARTICLE  COURS  SEC 101
%  utilise main-article et UStructure, UArticle
%========================================

% 				Gestion des Risques

%========================================

%************************************************************
% Chargement des variables du modèle
%************************************************************

\input{../Tex/Notes/Commons/config-notes.tex}

%========================================
\newcommand {\ukeywords}
%========================================
{%---------------------------------------------------------------------
Incidents, forensic, crise
}%

%************************************************************
% Chargement des variables dédiées à l'article
%************************************************************

%========================================
\newcommand {\utitle}
{%---------------------------------------------------------------------
Réagir : De l'incident de sécurité à la crise
}%---------------------------------------------------------------------

%========================================
\newcommand {\uabstract}
{%---------------------------------------------------------------------
C\edoc donne les grands principes de la gestion des incidents, et la conduite de gestion de crise.
}%---------------------------------------------------------------------


%************************************************************
%  variable définissant  le corps de l'article
%************************************************************

%===========================
\newcommand {\Ucontribute}
{
\input{../Tex/Contribute/contrib-gen.tex}
%\input{../Tex/Contribute/CYBERDEF-      contribs.tex}
\input{../Tex/Contribute/contribs.tex}
}

%========================================
\newcommand {\ubody}
{%---------------------------------------------------------------------
\input{../Tex/Chapters/chap-IncidMan-intro.tex}
\input{../Tex/Chapters/chap-IncidMan-csirt.tex}
\input{../Tex/Chapters/chap-IncidMan-crisis.tex}
}%---------------------------------------------------------------------

%************************************************************
% Chargement  du MODELE
%************************************************************

\umainload

